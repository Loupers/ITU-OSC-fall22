\documentclass[a4paper]{article}
\usepackage[margin=0.5cm,includeheadfoot,asymmetric]{geometry}
\usepackage[utf8]{inputenc} % Required for inputting international characters
\usepackage{hyperref}
\hypersetup{
    colorlinks=true, %set true if you want colored links
    linktoc=all,     %set to all if you want both sections and subsections linked
    linkcolor=blue,  %choose some color if you want links to stand out
}
\usepackage{titlesec}
\usepackage{fancyhdr}
\usepackage{lastpage}
\usepackage{listings}
\usepackage{color}

\definecolor{dkgreen}{rgb}{0,0.6,0}
\definecolor{gray}{rgb}{0.5,0.5,0.5}
\definecolor{mauve}{rgb}{0.58,0,0.82}

\lstset{frame=tb,
  language=C,
  aboveskip=3mm,
  belowskip=3mm,
  showstringspaces=false,
  columns=flexible,
  basicstyle={\small\ttfamily},
  numbers=none,
  numberstyle=\tiny\color{gray},
  keywordstyle=\color{blue},
  commentstyle=\color{dkgreen},
  stringstyle=\color{mauve},
  breaklines=true,
  breakatwhitespace=true,
  tabsize=3
}

\fancyhf{}
\rhead{Jan Piroutek - \textit{jpir@itu.dk}}
\chead{Exam solution}
\rfoot{\thepage\ of \pageref{LastPage}}
\renewcommand{\headrulewidth}{1.0pt}
\renewcommand{\footrulewidth}{1.0pt}
\pagestyle{fancy}

\newcommand\sectionS[1]{%
  \section*{#1}%
  \addcontentsline{toc}{section}{#1}}
\newcommand\subsectionS[1]{%
  \subsection*{#1}%
  \addcontentsline{toc}{subsection}{#1}}

\titleformat{\section}[block]{\Large\bfseries\filcenter}{}{1em}{}
\titleformat{\subsection}[hang]{\bfseries}{}{1em}{}

\begin{document}
\pagestyle{fancy}

%----------------------------------------------------------------------------------------
%	TITLE PAGE
%----------------------------------------------------------------------------------------

\begin{titlepage} % Suppresses displaying the page number on the title page and the subsequent page counts as page 1
	\newcommand{\HRule}{\rule{\linewidth}{0.5mm}} % Defines a new command for horizontal lines, change thickness here

	\center % Centre everything on the page
	
	%------------------------------------------------
	%	Headings
	%------------------------------------------------
	
	\textsc{\LARGE IT University of Copenhagen}\\[1.5cm] % Main heading such as the name of your university/college
	
	\textsc{\Large Operating systems and C - BSOPSYC1KU}\\[0.5cm] % Major heading such as course name
	
	%------------------------------------------------
	%	Title
	%------------------------------------------------
	
	\HRule\\[0.4cm]
	
	{\huge\bfseries Solution for exam, category SWU fall 2022}\\[0.4cm] % Title of your document
	
	\HRule\\[1.5cm]
	
	%------------------------------------------------
	%	Author(s)
	%------------------------------------------------
	
	% If you don't want a supervisor, uncomment the two lines below and comment the code above
	{\large\textit{Author}}\\
	Jan \textsc{Piroutek} % Your name

    {\large\textit{Email}}\\
	jpir@itu.dk

	
	%------------------------------------------------
	%	Date
	%------------------------------------------------
	
	\vfill\vfill\vfill % Position the date 3/4 down the remaining page
	
	{\large\today} % Date, change the \today to a set date if you want to be precise
	
	%------------------------------------------------
	%	Logo
	%------------------------------------------------
	
	%\vfill\vfill
	%\includegraphics[width=0.2\textwidth]{placeholder.jpg}\\[1cm] % Include a department/university logo - this will require the graphicx package
	 
	%----------------------------------------------------------------------------------------
	
	\vfill % Push the date up 1/4 of the remaining page
	
\end{titlepage}

\tableofcontents

\vspace{2cm}

\newpage

\sectionS{Question 1 - Data lab}
\subsectionS{A. Describe your implementation of \textit{howManyBits(x)}}
The easiest solution for this would be create mask, that isolates each bit of the number,
then take that bit and use the \textit{!} twice (\textit{!!}. Single \textit{!} gives 
1 if the number is only full of zeros, which would be if the bit, for which the mask was created
was set to 0. So for counting them I need to negate the value again, so I would get 
0 in this case and 1 otherwise. Then I can just sum those numbers together with \textit{+}
operator. \\
This solution is easy, but takes a lot of operations, so I needed to be smarter. \\
Let's start from broader perspective. If I want to represent the number in the twos complement
I'll always need one bit to basically say if the number is positive or negative (MSB). 
And what with the rest of the number. \\
Now lets think little bit more how twos complement works. If I have some number represented in
twos complement, then first bit of the number is giving the value we need to subtract
from the value stored in the next $x-1$ bits. Lets say I have two numbers $x$ and its absolute
value $y = |x|$. If $x$ is positive, then $y = x$ and for representing $x$ as twos complement.
I just need as much bits as is the highest position of bit set to one (e.g. $x = 7 = 0b111$ I need
3 bits), plus 1 as the leading 0 for MSB. \\
Lets think about case when $x$ is negative. Then I know the MSB is set to one, and the rest
of the number tells me, how much I have to subtract from $2^{position\_of\_MSB}$. Neat thing 
about twos complement is this thing tho, if I have some negative value $x$, the positive value
of $-x$ is also representable as twos complement, except for $0b10000\cdots$, where there are only
zeros after MSB one. What if I just add one to the negative number? \\
I will state this $\rightarrow$ if I increment negative number of twos complement by one, it will still
need same amount of bits for representation as the old one. 
\begin{itemize}
  \item{$-8 + 1 = -7$ and $-8 = 0b1000$; $-7 = 0b1001$}
  \item{$-3 + 1 = -2$ and $-3 = 0b101$; $-2 = 0b110$} 
  \item{$-1 + 1 = 0$ and $-1 = 0b1$; $ 0 = 0b0$}
\end{itemize}

\noindent This way I can just get the value of $x + 1$ and compute how many bits I need for the absolute
value $|x + 1|$ (x is still negative). How to easily get this value? Twos complement
is really nice to me. If I want value of $-x$ all I need to negate all bits and add 1.
But what I want is $(~x + 1) - 1 = ~x$. \\

\noindent So now I just need to get the absolute value of $x$ and compute the position of first bit, that
is set to one. Lets split it this into two parts. \\

\noindent{\large{Getting absolute value}} \\

\noindent First I need to know if the number is negative or positive. If $x$ is positive I want to 
get $x$, otherwise I want to get $~x$. First I isolate MSB with mask of $0x80 << 0x18$ (minimal number
explained more in next subquestion). Then I can reuse code of conditional function (one of the other
tasks). Conditional would be if $x$ after mask is 0 or not. If it is 0, the $x$ was positive, negative
otherwise. Then I can flip the condition, negate it and add 1. This way I will either get integer
consisting of all ones or all zeros. In the end I just use this condition as mask for $x$ or $~x$ 
and pick the desired one. One of them will be turned to all zeros, other will remain the same.
So I can just use $|$ to get $x$ I want. Now I have aquired the requested value. \\
 
\noindent{\large{Computing number of bits}} \\

\noindent Now when I have the positive value all I need to do is to get position of first 1. 
First a copy the first bit to the right of it, in $\log_{2}(32)$ steps. Starting from one
I shift all bits to the right by one and take $x | (x >> 0x01)$. This copies the most left one
to the next bit. Then I do the same but with $0x02$, that copies the leftest one and its copy from
last step to the next two bits. Then with $0x04$ and so on. \\
\noindent know I have copied the leftest one and I need to count how many bits I have in number.
This either can be slow or I can use divide and conquer technique. Each bit has information how
many ones are in him (one if set to one, zero if set to zero). Now what I do is to align bits, that
I want to add and use mask to separete odd and even bits. In other words I move even bits under
odd bits and add them together. Now I know how many bits are in the two consecutive bits. Then
I do the same for the pairs of bits. Align (shift by two this time) and mask them (I need
a different mask. First mask needed to mask even bits $\rightarrow 0b0101010101\cdots = 0x55555555$,
this needs to mask even pairs $\rightarrow 0b00110011\cdots = 0x33333333$.
Next for the pairs of four bits, 8bits and in the end 16bits. But because I didn't wanted to use
more mask than necessary, my code could have left some bits set in first part of the number. 
So I needed to mask the remaining bits, so it wouldn't return bigger numbers than $32$ (maximal
amount of ones in integer).
\\

\noindent In the end only thing I need is to add the 1 bit, that I have not acounted for earlier 
(the MSB of twos complement, saying if it is negative or positive)  \\

\newpage

\begin{lstlisting}
int howManyBits(int x) {
  int min = 0x80 << 0x18;
  int fives = 0x55 + (0x55 << 0x08) + (0x55 << 0x10) + (0x55 << 0x18);
  int threes = 0x33 + (0x33 << 0x08) + (0x33 << 0x10) + (0x33 << 0x18);
  int of = 0x0F + (0x0F << 0x08) + (0x0F << 0x10) + (0x0F << 0x18);
  int max_num = 0x3F;
  int cond;
  int neg_cond;
  int count;

  // if is x negative change it to its positive value
  // otherwise keep original positive value
  // 0x80000000 can stay the same, because it still needs all of bits used
  // izolate first bit
  cond = x & min;
  // from previous task
  neg_cond = !!cond;
  cond = ~(neg_cond) + 1;
  // why ~x and not ~x + 1
  // I know that in every twos complement I need one bit for basically the sign
  // not adding one when negating the number won't change it's size, if I remmember to
  // add that one bit back to result
  x = ((~x) & cond) | (x & ~cond);

  //copy the first 1 bit to the right
  x = x | (x >> 0x01);
  x = x | (x >> 0x02);
  x = x | (x >> 0x04);
  x = x | (x >> 0x08);
  x = x | (x >> 0x10);


  // count the number of 1s in x
  // this was taken from book Hacket's delight (I believe it's called)
  // Don't know if you can count it as solved but I don't know any other way
  // This is based on divide and conquer
  // each bits knows how many 1's are in it (either 1 or 0)
  // them you compute how many bits set have two next door bits
  // them 4, 8 and so on
  count = x + (~((x >> 0x01) & fives) + 1);
  count = (count & threes) + ((count >> 0x02) & threes);
  count = (count + (count >> 0x04)) & of;
  count = count + (count >> 0x08);
  count = count + (count >> 0x10);
  count = count & max_num;
  // don't forget to add 1
  return count + 1;
}
\end{lstlisting}


\subsectionS{B. Describe your implementation of \textit{tmin(void)}}
From definition of twos complement. the minimal number is composed of bits, where 
each bit is set to one. \\
The function should return minimal integer, so that means I need 32 bits, all set to one.
Largest number I can use, which has all bits set is $0x80$ ($-128$). Then I can use the
left shift ($<<$) operator in C, which moves bits to the left, direction to highest bit, and 
copies their value to the old place. So if I do $0x80 << 0x01$, shift 8 bits set to one to the
left by one place, I get 9 bits all set to one, if it doesn't overflow. Integer has 32 bit, so
to get them all set to one I just need to shift $0x80$ by $0x18$ (24) places. This gives me
integer with all bits set to one $\rightarrow$ smallest number, that fits into integer.

\begin{lstlisting}
int tmin(void) {
  return 0x80 << 0x18;
}
\end{lstlisting}

\newpage
\sectionS{Question 2 - Attack lab}
\subsectionS{A. What happens when the c3 (ret)
assembly instruction is executed? Does
anything in the stack change? }

Just shortly stack is a virtual memory data structure, that holds data for called functions.
Whenever procedure is called, the return address is pushed to the top of the stack and then
it allocates memory for that procedure (like variables and procedure parameters). Next to stack
there is a register called stack pointer. That is an address on stack, where the current running
programm is. \\
When the c3 ret is called, there should be return address present on the stack. The programs 
stack pointer points on to the point in memory where return address is. Ret pops the address
out of the stack and stores it into instruction pointer register. Poping something 
from the stack is equivalent of incrementing of the stack pointer. The return address is location
of next programm instruction, that should be executed after finishing the procedure. After the pop
the return address and other procedure parameters won't be on stack anymore $\rightarrow$
the stack memory of that function will be deallocated.
\subsectionS{B. What is a gadget farm? Describe an
example of how you use one in your
code.}

The computer can't read the human code, it needs to be translated into machine code. Machine
code is basically sequence of bytes, that can be executed by computer. Every instruction
can be encoded to the machine code. When we write some procedure, the whole thing get encoded
into sequence of bytes, encoding instructions we want to execute. 
Each byte is stored somewhere and has its own address. When we are executing code,
there is a register instruction pointer (further called IP), which points to current instruction
we want to execute. \\
Gadget is sequence of bytes encoding some instructions, ended by $0xc3$, which encodes ret
instruction. When we have a block of memory, that contains multiple of functions and their  
instructions. If in that block of memory are some gadgets we call that block of memory gadget farm.
\\
We don't need to call the whole function when using gadgets. We can just set the instruction
pointer to the address of the gadget, which we want to use. After executing instructions 
of the gadget, the ret function is called and as stated earlier, sets instruction pointer
to the address poped from the stack. This is usefull for us, when we want to use return oriented
programming to attack someone computer.  \\
Return oriented programming is a technique, when we overflow stack with addresses of gadgets 
we want to execute. \\
Shortly on stack overflow. Some functions in C have problems, when reading input from user.
If they store input on stack and don't check the length of input, the input can be longer
than allocated space for it. Than it is stored closer to the bottom of the stack, overwriting
it. We can use this to overwrite return addresses or insert some code we want to execute onto stack.
\\
When we are using return oriented programming, we overflow stack and store addresses of all gadgets
from gadget farm we want to use. Important part is to overwrite the current return address to 
address of first byte of the first gadget we want to use. Then when the programm continues, it 
executes first gadget after return. Because the gadget is ended with $0xc3$ instruction it looks
into stack and takes next return address and goes to that place, executing another gadget if there 
is one and so on.

\newpage
\sectionS{Question 3 - Malloc lab}
\subsectionS{A. Explain in detail your implementation of the \texttt{mm\_malloc} function.}
Lets start from heap structure. Heap consists of blocks. Each block has a header and footer.
Header and footer are both 1 \textit{WORD} long and contains information about size of the block and
information, if the block is allocated or free in format $(size | 0x01)$ for allocated blocks
and $(size | 0x0)$ for free blocks. If the block is free between header and footer, there is
information about previous and next free block. The information is address of the block.
Each address is 1 \textit{WORD} long, so the minimal length of block is 4 \textit{WORD}s. 
Thanks to header 
and footer I can traverse though blocks like throufh doubly linked linked list. Information
about previous and next free blocks creates an explicit free list. This list can be used to 
search faster for free space, when allocating memory. \\
Now get to memory allocation. First I ignore any request for memory of size 0 or smaller, those
requests don't make any sense. Then I need to align block. The block should be aligned so
their size is divisible by \textit{DWORD} (double \textit{WORD}). 
If size is smaller than \textit{DWORD} it should take space of at least 4 \textit{WORD}s, so  
in case this memory is freed later, I can put this space into explicit free list, where I need
2 \textit{WORD}s for pointers to other free blocks and two more for header and footer. 
If it is not smaller I just need to align it to be divisible by \textit{DWORD}. \\
After alignment I have size of block I really need to allocate, including header and footer.
Then I can go through free list. I start from first free block in the list, check its size and
if it is enough I return the block, otherwise I take pointer to next free block, from the body
of current free block and continue on him, until I find fitting block or I end up on the end of 
the list. If no block was sufficient I return \textit{NULL}. \\
{\large Case 1: I have found free block}\\
\\
Now I have enough space to allocate. Now I need to do two things. First I have to check, if the
block isn't too big, so I don't waste any space. If the size that would be unused, could be 
its own free block (is at least 4 \textit{WORD}s long), I set up header and footer of newly allocated
block. In other words I put information about size and allocation to the header and footer.
Then I delete this block from free list. Then I go to the end of footer of this block. I create
new header and footer with information about this block, like size and the fact, that it's free.
In the end I add this block to the beginnig of the free list. Otherwise I don't split anything
and just update header and footer information, that the block is allocated and remove it from
free list. \\
Now the block is allocted and I can return the pointer. \\

\noindent{\large Case 2: Finding free block returned \textit{NULL}}\\
If no block is found I need to extend the heap. I either extend by heap by some constant or
,if requested size is even bigger, by the requested size. Basically I just align the 
number of \textit{WORD}s I need to allocate, to be even and then I call \textit{mem\_sbrk}
to get that amount of memory, after that I just put information about size (and that the block
is free) to the header and footer of returned memory. In the end I try to coalesce the memory.
Because this block was added to the end of the heap, only two coalescing cases can happen, out
of 4 described in the book. Either the block of memory at the end of heap before extension
was free, then I join these two blocks, remove end block of heap before extension from free
list and add newly created block to the free list. Or the other case, the last block 
was full, so I just add this new block to the free list and no joining is happening.
\\ 
After this I have free block big enough to contain requested memory, so I do same process as
described before, about placing information to the header and potencial splitting of new block.

\subsectionS{B. What is pointer arithmetic? Describe how you use it in your version of mm.c}

When using arithmetic operations on pointers in C, the value of result scales based on 
type of the pointer. For example, let me have an array of integer and a pointer to the beginning
of the array $int *x$. Just to have it easy let starting address of  an array be equal to zero
$int* x = 0$. Normaly if I used arithmetic operation like $x++$, the result would be $1$. 
But because $x$ is a pointer to some integer, which has size of \textit{4 bytes}, it will
return result scaled to the size of integer $x++ == x + sizeof(int) == 4$, which is address of
next integer in array. \\
For any type the expresion $pointer_to_type_t + x$ will be expanded to 
$pointer_to_type + (x * sizeof(type_t))$. \\

\noindent{\large Use of pointer arithmetic in mm.c}
Not sure if in my case was this the use or more of a problem I needed to avoid (maybe both).
When moving through list of blocks I needed to look into header. I had pointer to the beginning
of the block, but header was one \textit{WORD} behind. So I needed to change type of pointer to
char ($sizeof(char) = 1B$, so I could change the address by exactly one \textit{WORD} and not by
$WORD * sizeof(typeOfPointer)$. This also applies for going to next/previous blocks in the list, 
when I need to compute the next pointer from size allocated in the header and don't want to
multiply it by data type. Other important use is in explicit free list I have, when I need
the address of the next or previous free block, same thing applies as for header and footer
of normal block.

\newpage

\newpage
\sectionS{Question 4 - Topics from the class}
\subsectionS{A. What is the difference between traps, faults 
and aborts in the context of interrupts?}



\subsectionS{B. What is the difference between an ephemeral 
and a well-known port? Give examples of
when either is used.} 

When two computers want to communicate with each other they need to know, where the data should flow.
Each computer on the network is identified by its IP address, that is how the connection between
those computers is made. Modern computer however maybe needs to create more communication channels.
To distinguish those channels, the ports are used. Ports can be either physical or software 
implemented. If it is physical it's a hole for a cable to be put in. Software implemented is software
that with packet takes a number of port and packets going to the same port resolves together. \\
Well-known port is a number of port, that is taken by some application. The application basically
owns that port number and noone else can use it. This port is allocated for the application and 
clients can be sure, that this application will run on that port and won't suddenly change.
This is usefull when we want for someone 
to be able to connect to my computer, for example to use some server running on it. Example of this
application can be either email protocols (smtp, pop3 etc.) or hypertext tranfer protocol (http) for
tranfering document, that are used as websites. \\
Ephemeral ports are ports, that are created to serve some purpose and deleted afterwards. This is 
used usually if client want to connect to some server with well-known port. Client gets assigned 
some ephemeral port by kernel. Then clients does its bussiness with server and when the 
connection is no longer neede, client releases the port. \\
The difference is that well-known ports are stable and represents one specific service and ephemeral
ports are short-lived and noone can be sure, which application is behind that port.

\subsectionS{C. What is a memory leak? When does it occur? 
What can you do to avoid it?} 

Memory leak is a problem of leaving allocated memory on heap, that will never be used again. 
Usually this happens, when programmer allocates some memory in function and then returns from 
it without freeing the memory or returning the pointer. In that case there will be allocated
memory, to which he lost pointer, so he can't free it. If this happens repeatedly, the memory
slowly fills up with the unfreed memory and in the worst case the whole computer runs
out of memory. This is really for applications, that should run all the time and never quit, like
servers. \\
To prevent this we (as programmers) always have to free allocated memory. 
\footnote{Or use language with garbage collector, that frees the memory for us, but I guess
this wasn't entirely point of this class}

\subsectionS{D. What is a race-condition? Why is a race-condition hard 
to debug? Which instructions can
you use to avoid race-conditions? 
Why are these instructions expensive?} 

Race condition is situation, when there are multiple threads running and one thread
depends on executing some instructions sooner than the other threads. For example 
one thread needs to read some variable, before other thread writes into it. The result
depends on who reaches the instruction first. \\
This is problem, when debugging. In theory this bug can occur only once and never be replicated
again. Or what is maybe worse, if there is multiple threads reaching same variable,
there will be multiple different outcomes each time, the program runs.\footnote{Not 
each time, because the number of outcomes is finite, so if we run it enough times
it will return the same result at some point.}
This makes it really hard to debug (not being able to replicate error). If some thread 
reaches desired variable first depends on, how the operating system schedules threads. \\
To prevent race-condition it's necessary to allow exclusive access to the part of code,
where the race-condition occurs. For this there is a semaphore. Basically it is just
one global variable, that is used by thread to communicate, if they can continue (go
though crossroad) or not. Before the thread reaches the critical part of the code,
we can let it waiting for getting access from semaphor. If the thread reaches this part
of code, it decrements the value of semaphore (the value is non-negative) and constantly
checks the variable until it is non-zero. If the variable is non-zero, it means green light
for thread, so it can access the critical section and proccess it. When leaving critical section 
thread increments value of semaphor variable, to give signal to other threads, saying that next 
one can access the critical section. But to be precise there is no rule saying, that this
variable has to be incremented by the same thread that is accessing critical area. The threads
waiting are waiting for signal for any other thread, not necessary the last one, that has accessed
the critical section. If we don't want this behaviour, there is a special type of sempahore called
mutex (mutual exclusion), where one thread gets access to shared resource and only that thread
can release it. Meaning other threads are waiting for the thread having the resource to tell
them, that it's ok to go.
\\
While waiting thread actively checks the variable every time it gets the chance. This 
requires some computational power and is a reason, why it is so expensive.


%----------------------------------------------------------------------------------------
\end{document}

