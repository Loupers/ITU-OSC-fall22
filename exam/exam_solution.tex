\documentclass[a4paper]{article}
\usepackage[margin=0.5cm,includeheadfoot,asymmetric]{geometry}
\usepackage[utf8]{inputenc} % Required for inputting international characters
\usepackage{hyperref}
\hypersetup{
    colorlinks=true, %set true if you want colored links
    linktoc=all,     %set to all if you want both sections and subsections linked
    linkcolor=blue,  %choose some color if you want links to stand out
}
\usepackage{titlesec}
\usepackage{fancyhdr}
\usepackage{lastpage}
\usepackage{listings}
\usepackage{color}

\usepackage[backend=biber]{biblatex}
\addbibresource{sample.bib}

\definecolor{dkgreen}{rgb}{0,0.6,0}
\definecolor{gray}{rgb}{0.5,0.5,0.5}
\definecolor{mauve}{rgb}{0.58,0,0.82}

\lstset{frame=tb,
  language=C,
  aboveskip=3mm,
  belowskip=3mm,
  showstringspaces=false,
  columns=flexible,
  basicstyle={\small\ttfamily},
  numbers=none,
  numberstyle=\tiny\color{gray},
  keywordstyle=\color{blue},
  commentstyle=\color{dkgreen},
  stringstyle=\color{mauve},
  breaklines=true,
  breakatwhitespace=true,
  tabsize=3
}

\fancyhf{}
\rhead{Jan Piroutek - \textit{jpir@itu.dk}}
\chead{Exam solution}
\rfoot{\thepage\ of \pageref{LastPage}}
\renewcommand{\headrulewidth}{1.0pt}
\renewcommand{\footrulewidth}{1.0pt}
\pagestyle{fancy}

\newcommand\sectionS[1]{%
  \section*{#1}%
  \addcontentsline{toc}{section}{#1}}
\newcommand\subsectionS[1]{%
  \subsection*{#1}%
  \addcontentsline{toc}{subsection}{#1}}

\titleformat{\section}[block]{\Large\bfseries\filcenter}{}{1em}{}
\titleformat{\subsection}[hang]{\bfseries}{}{1em}{}

\begin{document}

%----------------------------------------------------------------------------------------
%	TITLE PAGE
%----------------------------------------------------------------------------------------

\begin{titlepage} % Suppresses displaying the page number on the title page and the subsequent page counts as page 1
	\newcommand{\HRule}{\rule{\linewidth}{0.5mm}} % Defines a new command for horizontal lines, change thickness here

	\center % Centre everything on the page
	
	%------------------------------------------------
	%	Headings
	%------------------------------------------------
	
	\textsc{\LARGE IT University of Copenhagen}\\[1.5cm] % Main heading such as the name of your university/college
	
	\textsc{\Large Operating systems and C - BSOPSYC1KU}\\[0.5cm] % Major heading such as course name
	
	%------------------------------------------------
	%	Title
	%------------------------------------------------
	
	\HRule\\[0.4cm]
	
	{\huge\bfseries Solution for exam, category SWU fall 2022}\\[0.4cm] % Title of your document
	
	\HRule\\[1.5cm]
	
	%------------------------------------------------
	%	Author(s)
	%------------------------------------------------
	
	% If you don't want a supervisor, uncomment the two lines below and comment the code above
	{\large\textit{Author}}\\
	Jan \textsc{Piroutek} % Your name

    {\large\textit{Email}}\\
	jpir@itu.dk

	
	%------------------------------------------------
	%	Date
	%------------------------------------------------
	
	\vfill\vfill\vfill % Position the date 3/4 down the remaining page
	
	{\large\today} % Date, change the \today to a set date if you want to be precise
	
	%------------------------------------------------
	%	Logo
	%------------------------------------------------
	
	%\vfill\vfill
	%\includegraphics[width=0.2\textwidth]{placeholder.jpg}\\[1cm] % Include a department/university logo - this will require the graphicx package
	 
	%----------------------------------------------------------------------------------------
	
	\vfill % Push the date up 1/4 of the remaining page
	
\end{titlepage}

\tableofcontents

\vspace{2cm}

\newpage

\sectionS{Question 1 - Data lab}
\subsectionS{A. Describe your implementation of \textit{howManyBits(x)}}
The easiest solution for this would be to create a mask, that isolates each bit of the number,
then takes that bit and uses the \textit{!} twice (\textit{!!}). Single \textit{!} gives 
1 if the number is only full of zeros, which would be if the bit, for which the mask was created, 
was set to 0. So for counting them I need to negate the value again, so I would get 
0 in this case and 1 otherwise. Then I can add those numbers together with the \textit{+}
operator. \\
This solution is easy, but takes a lot of operations, so I needed to be smarter. \\
Let's start from a broader perspective. If I want to represent the number in the twos complement
I will always need one bit to say if the number is positive or negative (MSB). \\
And regarding the rest of the number, 
let's think a little bit more about how the twos complement works. If I have some number 
represented in
a twos complement, then the first bit of the number is giving the value we need to subtract
from the value stored in the next $x-1$ bits. Let's say I have two numbers $x$ and its absolute
value $y = |x|$. If $x$ is positive, then $y = x$ and for representing $x$ as twos complement
I just need as many bits as the highest position of a bit, that is set to 1 
(e.g. $x = 7 = 0b111$ I need 3 bits), plus 1 as the leading 0 for the MSB. \\
Let's think about a case when $x$ is negative. Then I know the MSB is set to 1, and the rest
of the number tells me, how much I have to subtract from $2^{position\_of\_MSB}$. Neat thing 
about the twos complement is this thing though, if I have some negative value $x$, the positive value
of $-x$ is also representable in the twos complement, except for $0b10000\cdots$, where there are only
zeros after MSB, that is set to 1. What if I just add 1 to the negative number? \\
I will state this $\rightarrow$ if I increment a negative number that is represented in the 
twos complement by 1, it will still
need the same amount of bits to be represented as the old one. 
\begin{itemize}
  \item{$-8 + 1 = -7$ and $-8 = 0b1000$; $-7 = 0b1001$}
  \item{$-3 + 1 = -2$ and $-3 = 0b101$; $-2 = 0b110$} 
  \item{$-1 + 1 = 0$ and $-1 = 0b1$; $ 0 = 0b0$}
\end{itemize}

\noindent This way I can just get the value of $x + 1$ and compute how many bits I need for the absolute
value $|x + 1|$ (x is still negative). How to easily get this value? The twos complement
makes this really simple. If I want the value of $-x$ all I need to do is negate all bits and add 1.
But what I want is $(~x + 1) - 1 = ~x$. \\

\noindent So now I just need to get the absolute value of $x$ and compute the position 
of the first bit, that
is set to 1. Lets split this into two parts. \\

\noindent{\large{Getting the absolute value}} \\

\noindent First I need to know if the number is negative or positive. If $x$ is positive I want to 
get $x$, otherwise I want to get $\sim x$. 
First I isolate the MSB with a mask of $0x80 << 0x18$ (minimal number
explained more in next subquestion). Then I can reuse the code from the conditional function 
(one of the other
tasks). Condition would be true if $x$ after mask is 0 or not. 
If it is 0 the $x$ was positive, negative
otherwise. Then I can flip the condition, negate it and add 1. This way I will either get an integer
consisting of all 1s or all 0s. In the end I just use this condition as a mask for $x$ or $\sim x$ 
and pick the desired one. One of them will be turned to all zeros, the other will remain the same.
So I can just use the $|$ operator to get the value I want. Now I have aquired the requested value. \\
 
\noindent{\large{Computing the number of bits}} \\

\noindent Now when I have the positive value all I need to do is to get the position of the first 1. 
First I need is to have 1s on every position after the first 1, that was found. 
I can achieve than in $\log_{2}(32)$ steps. Starting from 1
I shift all bits to the right by $0x01$ and take the value of $x | (x >> 0x01)$. 
This copies the most left one
to the next bit. Then I do the same but with $0x02$, that copies the leftest one and its copy from
last step to the next two bits. Then with $0x04$ and so on, until I have copied it to the LSB.  \\
\noindent Now I have copied the leftest 1 and I need to count how many bits I have set in the number.
I can either do it slowly or I can use divide and conquer technique. Each bit has information how
many 1s are in him (one if set to 1, zero if set to 0). Now what I do is to align bits, that
I want to add and use mask to separate odd and even bits. In other words I move even bits under
odd bits and add them together. Now I know how many bits are in the pairs of the consecutive bits. 
Then I do the same for the pairs of bits. Align (shift by two this time) and mask them (I need
a different mask. First mask I needed to mask even bits $\rightarrow 0b0101010101\cdots = 0x55555555$,
this needs to mask even pairs $\rightarrow 0b00110011\cdots = 0x33333333$.
Next for the pairs of 4 bits, 8 bits and in the end 16 bits. But because I didn't wanted to use
more masks than necessary, my code could have left some bits set in the first part of the number. 
I needed to mask the remaining bits, so it wouldn't return bigger numbers than $32$ (maximal
amount of 1s in an integer).
\\

\noindent In the end the only thing I need is to add the 1 bit, that I have not acounted for earlier 
(the MSB of twos complement, saying if it is negative or positive).  \\

\newpage

\begin{lstlisting}
int howManyBits(int x) {
  int min = 0x80 << 0x18;
  int fives = 0x55 + (0x55 << 0x08) + (0x55 << 0x10) + (0x55 << 0x18);
  int threes = 0x33 + (0x33 << 0x08) + (0x33 << 0x10) + (0x33 << 0x18);
  int of = 0x0F + (0x0F << 0x08) + (0x0F << 0x10) + (0x0F << 0x18);
  int max_num = 0x3F;
  int cond;
  int neg_cond;
  int count;

  // if is x negative change it to its positive value
  // otherwise keep original positive value
  // 0x80000000 can stay the same, because it still needs all of bits used
  // izolate first bit
  cond = x & min;
  // from previous task
  neg_cond = !!cond;
  cond = ~(neg_cond) + 1;
  // why ~x and not ~x + 1
  // I know that in every twos complement I need one bit for basically the sign
  // not adding one when negating the number won't change it's size, if I remmember to
  // add that one bit back to result
  x = ((~x) & cond) | (x & ~cond);

  //copy the first 1 bit to the right
  x = x | (x >> 0x01);
  x = x | (x >> 0x02);
  x = x | (x >> 0x04);
  x = x | (x >> 0x08);
  x = x | (x >> 0x10);


  // count the number of 1s in x
  // this was taken from book Hacket's delight (I believe it's called)
  // Don't know if you can count it as solved but I don't know any other way
  // This is based on divide and conquer
  // each bits knows how many 1's are in it (either 1 or 0)
  // them you compute how many bits set have two next door bits
  // them 4, 8 and so on
  count = x + (~((x >> 0x01) & fives) + 1);
  count = (count & threes) + ((count >> 0x02) & threes);
  count = (count + (count >> 0x04)) & of;
  count = count + (count >> 0x08);
  count = count + (count >> 0x10);
  count = count & max_num;
  // don't forget to add 1
  return count + 1;
}
\end{lstlisting}


\subsectionS{B. Describe your implementation of \textit{tmin(void)}}
From definition of the twos complement, the minimal number is composed of bits, where 
the MSB is set to 1 and all other bits are set to 0. \\
The function should return minimal integer, so that means I need 32 bits.
The longest minimal number I can use is $0x80$ ($-128$). Then I can use the
left shift ($<<$) operator in C, which moves bits to the left, direction to MSB, and 
the places, where the original number was shifted from, will be set to 0. 
So if I do $0x80 << 0x01$, it shifts 8 bits to the 
left by one place. I get 9 bits, first set to 1, others to 0, if it doesn't overflow. 
Integer has 32 bit, so
to get them all set the way I want, I just need to shift $0x80$ by $0x18$ (24) places. This gives me
the integer with MSB set to one and all other bits set to 0 $\rightarrow$ smallest number, 
that fits into an integer.

\begin{lstlisting}
int tmin(void) {
  return 0x80 << 0x18;
}
\end{lstlisting}

\newpage
\sectionS{Question 2 - Attack lab}
\subsectionS{A. What happens when the c3 (ret)
assembly instruction is executed? Does
anything in the stack change? }

Just shortly,the call stack is a virtual memory data structure that holds data for procedures, 
that has been called.
Whenever a procedure is called, the return address is pushed to the top of the stack and then
it allocates a memory for the procedure (like variables and procedure parameters). Next to the stack
there is a register called stack pointer. That is an address on the stack, where the current running
programm is. \\
When the c3 ret is executed, there should already be a return address present on the stack. 
The programs 
stack pointer points to the point in the memory where the return address is. Ret pops the address
out of the stack and stores it in the instruction pointer register. Poping something 
from the stack is equivalent to incrementing the stack pointer. The return address is location
of the next program instruction, that should be executed after finishing the procedure. After the pop
the return address and other procedure parameters won't be on stack anymore $\rightarrow$
the stack memory of that function will be deallocated.
\subsectionS{B. What is a gadget farm? Describe an
example of how you use one in your
code.}

The computer can't read the human code, it needs to be translated to the machine code. Machine
code is basically a sequence of bytes, that can be executed by a computer. Every instruction
can be encoded to the machine code. When we write some procedure, the whole thing gets encoded
to a sequence of bytes, encoding instructions we want it to execute. 
Each byte is stored somewhere in the memory and has its own address. When we are executing a code,
there is a register instruction pointer, which points to current instruction
we want to execute. \\
Gadget is a sequence of bytes encoding some instructions. The last byte of the gadget
has to be $0xc3$ byte, which encodes the ret
instruction. When we have a block of some memory, that contains multiple functions and their  
instructions. If in the block of the memory are some gadgets we call that block of memory a 
gadget farm.
\\
We don't need to call the whole function when using gadgets. We can just set the instruction
pointer to the address of the gadget, that we want to use. After executing instructions 
of the gadget, the ret function is called and sets the instruction pointer
to the address poped from the stack. This is usefull for us, when we want to use return oriented
programming to attack someones computer.  \\
Return oriented programming is a technique, when we overflow stack with addresses of gadgets 
we want to execute. \\
Some functions in C have problems, when reading input from the user.
If they store the input on the stack and don't check the length of the input, the stored input can 
be longer
than the allocated space for it. Than it will be stored closer to the bottom of the stack, overwriting
already stored data on it. We can use this to overwrite return addresses or insert some code 
we want to execute on the stack. This is called stack overflow.
\\
When we are using return oriented programming, we overflow the stack and store addresses 
of all gadgets
from the gadget farm we want to use. Important part is to overwrite the current return address to 
the address of the first byte of the the first gadget we want to use. 
Then when the program continues, it 
executes the first gadget after return. Because the gadget ends with the $0xc3$ instruction, it looks
into the stack, takes the next return addres and saves it in the instruction pointer. 
Instruction pointer then executes the code of the next gadget and the process repeats, as long
there are valid instruction addresses on the stack.

\newpage
\sectionS{Question 3 - Malloc lab}
\subsectionS{A. Explain in detail your implementation of the \texttt{mm\_malloc} function.}
Lets start from the heap structure. The Heap consists of blocks. Each block has a header and a footer.
Header and footer are both 1 \textit{WORD} long and contains information about size of the block and
information, if the block is allocated or free. 
That information is stored in format $(size | 0x01)$ for allocated blocks
and $(size | 0x0)$ for free blocks. If the block is free between header and footer, there is
information about previous and next free block. The information is address of the block.
Each address is 1 \textit{WORD} long, so the minimal length of block is 4 \textit{WORD}s. 
Thanks to the header 
and the footer I can traverse blocks like a doubly linked linked list. Information
about previous and next free blocks creates an explicit free list. This list can be used to 
search faster for free space, when allocating memory, because we don't need to check all blocks,
but just the free ones.\\
\begin{lstlisting}[caption= Manipulating with the explicit free list]
// put bp to beginning of the free list
static void add_to_free_list(void* bp) {
	NEXT_FREE_PTR(bp) = exp_free_listp;
	PREV_FREE_PTR(exp_free_listp) = bp;
	PREV_FREE_PTR(bp) = NULL;
	exp_free_listp = bp;
}

// delete block from free list
static void delete_from_free_list(void * bp) {
	if (PREV_FREE_PTR(bp) == NULL) {
		// bp points to beginnig of free list
		exp_free_listp = NEXT_FREE_PTR(bp);
	} else {
		//void *p = PREV_FREE_PTR(bp);
		//void *n = NEXT_FREE_PTR(bp);
		NEXT_FREE_PTR(PREV_FREE_PTR(bp)) = NEXT_FREE_PTR(bp);
	}
	PREV_FREE_PTR(NEXT_FREE_PTR(bp)) = PREV_FREE_PTR(bp);
}
\end{lstlisting}

As you can see in the code listing above I have used a lot of macros to simplify
pointer manipulation for me. Most of the macros I have used are taken from
the course book \cite{c1}. Only one defined by me are \textit{NEXT\_FREE\_PTR}
and \textit{PREV\_FREE\_PTR}.

\begin{lstlisting}[caption= Defined macros]
/* Pack a size and allocated bit into a word */
#define PACK(size, alloc) ((size) | (alloc))

/* Read and write a word at address p */
#define GET(p) (*(unsigned int *)(p))
#define PUT(p, val) (*(unsigned int *)(p) = (val))

/* Read the size and allocated fields from address p */
#define GET_SIZE(p) (GET(p) & ~0x07)
#define GET_ALLOC(p) (GET(p) & 0x1)

/* Given block ptr bp, compute address of its header and footer */
#define HDRP(bp) ((char *)(bp) - WSIZE)
#define FTRP(bp) ((char *)(bp) + GET_SIZE(HDRP(bp)) - DSIZE)

/* Given block ptr bp, compute address of next and previous blocks */
#define NEXT_BLKP(bp) ((char *)(bp) + GET_SIZE(HDRP(bp)))
#define PREV_BLKP(bp) ((char *)(bp) - GET_SIZE((char *)(bp) - DSIZE))

// jumping through free list
#define NEXT_FREE_PTR(bp) (*(char **)(bp + WSIZE))
#define PREV_FREE_PTR(bp) (*(char **)(bp))
\end{lstlisting}

Now to the memory allocation. First I ignore any nonpositive request for a memory, those
requests don't make any sense. In that case I return \textit{NULL} pointer. 
Then I need to align the size of the block. The block should be aligned so
its size is divisible by \textit{DWORD} ($2 * WORD$). 
If the size is smaller than \textit{DWORD} it should take space of at least 4 \textit{WORD}s. 
That means, that In case this memory is freed later, I can put this space in the explicit 
free list, where I need
2 \textit{WORD}s for pointers to other free blocks and two more for the header and the footer. 
If the size is not smaller I just need to align it to be divisible by \textit{DWORD}. \\
After the alignment I have the size of the block I really need to allocate, 
including size for a header and a footer.
I have created a function, that help me go through the free list to find some satisfying block. 
I start from 
the first free block in the list, check its size and
if it is enough I return the block, otherwise I take the pointer to the next free block, 
from the body of the current free block and continue on him, until I find a fitting block 
or I end up on the end of the list. If there wasn't any sufficient block I return \textit{NULL}
from the function. \\

\begin{lstlisting}[caption= Finding free block in the explicit free list]
// find place for memory of size 'size'
static void* find_fit(size_t size) {
	// first fit algorithm for implicit free list
	void *bp;
	for (bp = exp_free_listp; GET_ALLOC(HDRP(bp)) == 0; bp = NEXT_FREE_PTR(bp)) {
		if (size <= GET_SIZE(HDRP(bp))) {
			return bp;
		}
	}
	return NULL;
}
\end{lstlisting}

{\large Case 1: I have found free block}\\
\\
After finding enough space for allocation, I need to do two things. First I have to check, if the
block isn't too big, because I don't want to waste any space. 
If the unused size could be 
its own free block (is at least 4 \textit{WORD}s long), I set up the header and the footer 
of the newly allocated
block to contain its size and information about being allocted. 
Then I delete this block from the explicit free list. 
Now I go to the end of the footer of this block. I create
a new header and footer with information about this block, like size and the fact, that it's free.
In the end I add this block to the beginnig of the explicit free list. 
If the unused size of the block would be enough, I don't split anything
and just update header and footer information, that the block is allocated and remove it from
the explicit free list. 
\begin{lstlisting}[caption= {helper function for setting a block as allocated}]
// set header and footer of newly allocated block
static void place(void* bp, size_t size) {
	// get size of while block
	size_t csize = GET_SIZE(HDRP(bp));

	// if the size of block is bigger than requested size
	// and after allocation there will still be enough space for header and footer (and payload)
	// split the block
	if ((csize - size) >= (2*DSIZE)) {
		// put info to block header that blocks size is 'size' and 1 for allocated
		// same for footer
		PUT(HDRP(bp), PACK(size, 1));
		PUT(FTRP(bp), PACK(size, 1));
		delete_from_free_list(bp);
		// move pointer to end of needed space for allocation
		bp = NEXT_BLKP(bp);
		// here the block splits
		// put info to start of new block about its size (csize - size) and info that it's free 0
		PUT(HDRP(bp), PACK(csize-size, 0));
		PUT(FTRP(bp), PACK(csize-size, 0));
		//add_to_free_list(bp);
		coalesce(bp);
	} else {
		// just allocate whole block
		PUT(HDRP(bp), PACK(csize, 1));
		PUT(FTRP(bp), PACK(csize, 1));
		delete_from_free_list(bp);
	}
}
\end{lstlisting}
Now the block is allocted and I can return the pointer to it. \\
\newpage

\noindent{\large Case 2: Finding free block returned \textit{NULL}}\\
If no free block is found I need to extend the heap. I either extend by heap by some constant or
if requested size is even bigger than that constant by the requested size. Because the supporting 
function for heap extension requires number of \textit{WORD}, I will have to work 
with $size \ WORD_SIZE$. 
For extending I need to align the 
number of \textit{WORD}s I am allocating to be even and then I call provided function 
\textit{mem\_sbrk}
to extend the heap by that amount of \textit{WORD}s. After that I put information 
about size (and that the block
is free) to the header and footer of the returned memory.
\begin{lstlisting}[caption= {function for extending heap}]
// extending heap
static void *extend_heap(size_t words)
{
	void *bp;
	size_t size;

	/* Allocate an even number of words to maintain alignment */
	size = (words % 2) ? (words+1) * WSIZE : words * WSIZE;
	if ((long)(bp = mem_sbrk(size)) == -1)
		return NULL;

	// Initialize free block header/footer and the epilogue header
	PUT(HDRP(bp), PACK(size, 0));
	PUT(FTRP(bp), PACK(size, 0));
	// last block is of size 0
	PUT(HDRP(NEXT_BLKP(bp)), PACK(0, 1));
	// merge if last block was free
	return coalesce(bp);
}
\end{lstlisting}

In the end I try to coalesce the memory.
Because this block was added to the end of the heap, only two out of four coalescing cases can happen.
Either the block of the memory at the end of the heap before extension
was free, then I have to join these two blocks. First I need to check the header of the last 
block on the heap to see, if the block is free. If it is not free, I just need to add the  
new block to the free list (the block, that was returned by \textit{mem\_sbrk}).
If the block was free, first thing I need to do is remove that block from the free list.
Then I need to update the size of the block, that will be created, to sum of sizes of those
two blocks. I put information about size and being free to the header of the last block on the heap
and to the footer of the new block returned by \textit{mem\_sbrk}. Then I add this new joined
block to the explicit free list and that's it.

\begin{lstlisting}[caption= {coalescing function, important are cases 1 and 3}]
static void *coalesce(void *bp)
{
	// is previous block allocated or is bp the first block in the heap
	size_t prev_alloc = GET_ALLOC(FTRP(PREV_BLKP(bp))) || PREV_BLKP(bp) == bp;
	// is next block allocated
	size_t next_alloc = GET_ALLOC(HDRP(NEXT_BLKP(bp)));
	// size of current block
	size_t size = GET_SIZE(HDRP(bp));

	// 4 cases for surrounding blocks
	if (prev_alloc && next_alloc) {
		/* Case 1 */
		// everything around block is allocated
		// can't be merged with anything
		add_to_free_list(bp);
		return bp;
	}
	else if (prev_alloc && !next_alloc) {
		/* Case 2 */
		// previous block is allocated and next block is free
		// merge this block with next block
		
		
		// increase total size with size of next block
		size += GET_SIZE(HDRP(NEXT_BLKP(bp)));
		// delete next block from free list so I can add merged block later
		delete_from_free_list(NEXT_BLKP(bp));

		// put info to blocks header about new size and that it's free
		// also into footer
		PUT(HDRP(bp), PACK(size, 0));
		PUT(FTRP(bp), PACK(size, 0));
	}
	else if (!prev_alloc && next_alloc) {
		/* Case 3 */
		// previous block is free, but next block is allocated
		// merge this block with previous block
		
		// increase total size with size of previous block
		size += GET_SIZE(HDRP(PREV_BLKP(bp)));
		// delete previous block from free list so I can add merged block later
		delete_from_free_list(PREV_BLKP(bp));

		// put info to footer of this block about size and free 0
		PUT(FTRP(bp), PACK(size, 0));
		// put info to header of previous block
		PUT(HDRP(PREV_BLKP(bp)), PACK(size, 0));
		// set pointer to previous block
		bp = PREV_BLKP(bp);
	}
	else {
		/* Case 4 */
		// both previous and next block are free
		// have to merge them all together
		
		// total size is sum of size of this block + previous block + next block
		size += GET_SIZE(HDRP(PREV_BLKP(bp))) +
			GET_SIZE(FTRP(NEXT_BLKP(bp)));

		// delete both previous and next block from free list so merged can be added later
		delete_from_free_list(PREV_BLKP(bp));
		delete_from_free_list(NEXT_BLKP(bp));
		// put info about size and free to header of previous block
		PUT(HDRP(PREV_BLKP(bp)), PACK(size, 0));
		// put info about size and free to footer of next block
		PUT(FTRP(NEXT_BLKP(bp)), PACK(size, 0));
		// set pointer to previous block
		bp = PREV_BLKP(bp);
	}
	// add merged block
	add_to_free_list(bp);
	return bp;
}
\end{lstlisting}

After this I have a free block big enough to contain the requested memory, so I do same process as
described before, about placing information to the header and potencial splitting of new block.

\begin{lstlisting}[caption= {implementation of malloc function}]
/*
 * mm_malloc - Allocate a block by incrementing the brk pointer.
 *     Always allocate a block whose size is a multiple of the alignment.
 */
void *mm_malloc(size_t size)
{
	size_t asize;
	/* Adjusted block size */
	size_t extendsize; /* Amount to extend heap if no fit */
	void *bp;

	/* Ignore spurious requests */
	if (size == 0)
		return NULL;
	/* Adjust block size to include overhead and alignment reqs. */
	if (size <= DSIZE)
		asize = 2*DSIZE;
	else
		asize =  (DSIZE) * ((size + (DSIZE) + (DSIZE-1)) / DSIZE);

	/* Search the free list for a fit */
	if ((bp = find_fit(asize)) != NULL) {
		place(bp, asize);
		return bp;
	}

	/* No fit found. Get more memory and place the block */
	extendsize = MAX(asize,CHUNKSIZE);
	if ((bp = extend_heap(extendsize/WSIZE)) == NULL)
		return NULL;

	place(bp, asize);

	return (void *)bp;
}
\end{lstlisting}

\subsectionS{B. What is pointer arithmetic? Describe how you use it in your version of mm.c}

When using arithmetic operations on pointers in C, the value of the result scales based on the
type of the pointer. For example, let me have an array of integers and a pointer to the beginning
of the array $int *x$. Just to have it easy let starting address of the array be equal to zero
$int* x = 0$. Normaly if I used arithmetic operation like $x++$ for x being normal integer, 
the result would be $1$. 
But because $x$ is a pointer to an integer, which has a size of \textit{4 bytes}, it will
return result scaled to the size of an integer $x++ == x + sizeof(int) == 4$, which is the address of
the next integer in the array. \\
For any type the expresion $pointer\_to\_type\_t + x$ will be expanded to 
$pointer\_to\_type\_t + (x * sizeof(type\_t))$. \\

\noindent{\large Use of the pointer arithmetic in mm.c}
Not sure if in my case was the pointer arithmetic usefull or more of a problem 
I needed to avoid (maybe both).
When moving through a list of blocks I needed to look into the header of the block. 
I had the pointer to the beginning
of the block, but the header was one \textit{WORD} behind. So I needed to change 
the type of the pointer to
a char ($sizeof(char) = 1B$, so I could change the address by exactly one \textit{WORD} and not by
$WORD * sizeof(typeOfPointer)$. If I haven't done that and the pointer I got was of a type of
the integer, I would have skipped the header and looked somewhere into the memory of the previous
block. I also needed to think about this when going to the next/previous blocks 
in the explicit free list, 
when I needed to compute the next pointer from the size located in the header and didn't want to
multiply it by the pointer type. Other important use is in the explicit free list. When I need
the address of the next or previous free block, the same problem as described above applies here.

\newpage
\sectionS{Question 4 - Topics from the class}
\subsectionS{A. What is the difference between traps, faults 
and aborts in the context of interrupts?}

From those three exception classes the trap is most similiar to interrupts I would say, 
especially when it 
comes to handling them. When a trap occures, the trap handler is called and after handling that trap,
the running of the program resumes by going to the next instruction in order. Traps happens,
when something is wrong with the call to the system function from the program. 
That's different from interrupt, they
occur, when there's something wrong with external devices. Interupt handler interupts current 
running program, handles the interupt and resumes program back by running the next instruction.
The aborts are exceptions, from which 
the program can't recover. Interrupts takes control from the application and then returns it
to the next instruction. Abort doesn't return control back to the program, but shuts it down.
When fault happens, there is possibility, that user won't notice it happened, same 
as with an interrupt. 
Interrupts are handled in the way, the program doesn't notice, they happened. When fault
comes, the fault handler tries to recover from it, which might be possible. If the recovery
is successful, fault handler returns to the current instruction of the program and executes
it again. But if there
was no way, to recover, fault handler acts like an abort and shuts the program down.

\subsectionS{B. What is the difference between an ephemeral 
and a well-known port? Give examples of
when either is used.} 

When two computers want to communicate with each other they need to know, the way the 
data should flow.
Each computer on the network is identified by its IP address, that is how the connection between
those computers is made. Modern computer however may need to create more communication channels.
To distinguish those channels, the ports are used. Port can be either physical or software 
implemented. If it is physical it's a hole for a cable to be put in. Software implemented is software
that with packet takes a number of port, to which the packet has arrived, 
and packets going to the same port resolves together. The channel between two computers identified
by their IP addresses and ports is called socket.\\
Well-known port is identified by a number, that is taken by some application. 
The application 
owns that port number and noone else can use it. This port is allocated for the application and 
clients can be sure, that this application will run on that port on that computer 
and won't suddenly change its port.
This is usefull when we want for someone 
to be able to connect to my computer, for example to use some server running on it. Example of this
application can be either email protocols (smtp, pop3 etc.) or hypertext tranfer protocol (http) for
tranfering document, that are used as websites. The well-known port is in use as long the application
is running.\\
Ephemeral ports are ports, that are created to serve some purpose and then deleted afterwards. 
This is 
used usually if a client want to connect to some server with well-known port. Client gets assigned 
some ephemeral port by the operating system kernel. Then client does its bussiness 
with that server and when the 
connection is no longer in need, the client releases the port. \\
The difference is that well-known ports are stable and represents one specific service running 
behind them and ephemeral
ports are short-lived and noone can be sure, which application is using that port.

\subsectionS{C. What is a memory leak? When does it occur? 
What can you do to avoid it?} 

Memory leak is a problem of leaving allocated memory on the heap, that will never be used again. 
Usually this happens, when a programmer allocates some memory in a function and then returns from 
that function without freeing the allocated memory or returning the pointer to that memory. 
In that case there will be allocated
memory on the heap, to which he lost pointer, so he can't free it later. 
If this happens repeatedly, the heap memory
slowly fills up with unfreed memory and in the worst case the whole computer runs
out of the memory. This is really big problem for applications, that should run all the time 
and never stop, like servers. \\
To prevent this we (as programmers) always have to free allocated memory,
or use some programming language with a garbage collector. Garbage collector is a dynamic memory
allocator, that is also responsible for freeing block, that are no longer neede by the application.
It does this automatically, so the programmer doesn't need to think about freeing allocated
memory.

\subsectionS{D. What is a race-condition? Why is a race-condition hard 
to debug? Which instructions can
you use to avoid race-conditions? 
Why are these instructions expensive?} 

Race condition is a situation, when there are multiple threads running and one thread
depends on executing some instructions sequentially without other threads interfering.
For example if threads needs to work for some time with a shared varible, that is 
shared to other threads, and there is an other thread, that changes the variable in
the middle of running computations of the first thread. The result then depends on which
thread is executed first and when it runs its code. If the first thread finishes before
other thread changes the variable, it will have different result, that when the other thread
changes variable in the middle of the computation of the first thread  or if the
other thread changes it before the first thread starts running the computations. This
situation is called race condition.\\
This is a problem, when debugging. In theory this bug can occur only once and never be replicated
again. Or what can be even worse, if there is multiple threads reaching same variable,
there will be different outcome each time, the program runs.\footnote{Not 
each time, because the number of outcomes is finite, so if we run it enough times
it will return the same result at some point.}
Not being able to replicate error makes it really hard to debug. If some thread 
reaches desired variable first depends on, how the operating system schedules threads. \\
To prevent race-condition it's necessary to allow exclusive access to the shared content.
For this there is a semaphore. Semaphore is just
one global variable, that is used by threads to communicate, if they can continue (go
though crossroad) or not. Before the thread reaches the critical part of the code
(manipulating with shared content),
we can let it wait for a signal to continue from the semaphor. The global variable representing
semaphore is just a non-negative integer. If the thread reaches this part
of code, it decrements the value of semaphore and constantly
checks the variable until it is non-zero. If the variable is not zero, it means green light
for the thread, so it can access the critical section and run it. When leaving the critical section 
the thread increments the value of the semaphor variable, to give a signal to other threads, 
saying that next 
one can access the critical section. But to be precise there is no rule saying, that this
variable has to be incremented by the same thread that is accessing the critical area. The threads
waiting are waiting for a signal for any other thread, not necessary the last one, that has accessed
the critical section. So we can have a thread that is incrementing the semaphore and allowing
other threads access shared content. So we can have more threads accessing the shared content. 
If we don't want this behaviour, there is a special type of a sempahore called
mutex (mutual exclusion), where one thread gets access to the shared resource and only that thread
can release (increment the semaphore variable) it. Meaning other threads are waiting for 
the thread having the resource to tell them, that it's ok to continue.
\\
While waiting thread actively checks the semaphore variable every time it gets the chance. This 
requires computational power and is a reason, why it is so expensive, because you need to run 
instructions just for the checking the variable.

\newpage
\printbibliography

\end{document}

